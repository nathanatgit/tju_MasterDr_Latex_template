%% Define general packages.
\usepackage{amsmath}
\usepackage{amssymb}        % math environment
\usepackage{graphicx}
\graphicspath{{figures/}}   % graphics environment
\usepackage{titlesec}       % redefine title style
\usepackage{titletoc}       % redefine contents list
\usepackage{multirow}       % table multi column and rows support
\usepackage{tabularx}       % advanced table environment
\usepackage{booktabs}       % table out line enhancement
\usepackage{emptypage}      % remove page numbers in blank page
% \usepackage{showframe}    % for detecting margins

% 位论文一律采用ISO 216标准规定的A4纸张(大小为210×297mm)。页边距为:上:27.5mm;下25.4mm;左:35.7mm;右:27.7mm。页眉距边界15.0mm;页脚距边界17.5mm。
% \usepackage[top= 27.5mm,bottom=25.4mm,left=27.7mm,right=27.7mm,head=15mm,foot=17.5mm,bindingoffset=8mm]{geometry}
\usepackage[top= 27.5mm,bottom=25.4mm,left=35.7mm,right=27.7mm,head=15mm,foot=17.5mm]{geometry}


% 正文中文字体为宋体,英文字体为Times New Roman,正文采用小四号字,段落行间距为固定值20磅,段落前后间距为0,首行缩进2字符。
% 西文字体以Times New Roman为准,若Times New Roman中没有相应字符,则应使用较为清晰和通用的字体。数学公式和专门文字(如计算机程序代码)的字体可以根据需要选择。
\usepackage[PunctStyle=kaiming,AutoFakeBold=true,AutoFakeSlant=true]{xeCJK}
%% set font family
\setCJKmainfont[Mapping=tex-text]{SimSun-ExtB}
\setmainfont{Times New Roman}

\usepackage[UTF8]{ctex}     % Chinese fonts support (like /songti /heiti)
\usepackage{fontspec}
\usepackage{setspace}       % define spaces between lines/words.
\usepackage{fancyhdr}       % redefine header and footer styles
\usepackage[unicode,colorlinks=ture,hidelinks,urlcolor=black,breaklinks=true]{hyperref} % add hyperlinks to contents, citations and cross reference of tables, equations, etc.

\usepackage{gbt7714}        % Chinese GBT 7714 style reference list
\bibliographystyle{gbt7714-numerical}

\usepackage{subfig}
\usepackage{caption}
\usepackage{array}
