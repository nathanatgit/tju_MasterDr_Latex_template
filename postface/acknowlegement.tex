\addcontentsline{toc}{chapter}{致谢}
\chapter*{致谢}
\markboth{致\ 谢}{致\ 谢}
\thispagestyle{fancy}

《水调歌头·明月几时有》·苏轼 〔宋代〕。丙辰中秋,欢饮达旦,大醉,作此篇,兼怀子由。

明月几时有?把酒问青天。不知天上宫阙,今夕是何年。我欲乘风归去,又恐琼楼玉宇,高处不胜寒。起舞弄清影,何似在人间。
转朱阁,低绮户,照无眠。不应有恨,何事长向别时圆?人有悲欢离合,月有阴晴圆缺,此事古难全。但愿人长久,千里共婵娟。

丙辰年(公元1076年)的中秋节,通宵痛饮直至天明,大醉,趁兴写下这篇文章,同时抒发对弟弟子由的怀念之情。

像中秋佳节如此明月几时能有?我拿着酒杯遥问苍天。不知道高遥在上的宫阙,现在又是什么日子。我想凭借着风力回到天上去看一看,又担心美玉砌成的楼宇太高了,我经受不住寒冷。起身舞蹈玩赏着月光下自己清朗的影子,月宫哪里比得上人间烟火暖人心肠。
月儿移动,转过了朱红色的楼阁,低低地挂在雕花的窗户上,照着没有睡意的人。明月不应该对人们有什么怨恨吧,可又为什么总是在人们离别之时才圆呢?人生本就有悲欢离合,月儿常有圆缺,(想要人团圆时月亮正好也圆满)这样的好事自古就难以两全。只希望这世上所有人的亲人都能平安健康长寿,即使相隔千里也能共赏明月。