%%% define controls and styling

%% Define Chinese control sequences for font sizes (temporarily use). Remove unnecessary if you don't need.
% the first means the font size, the second means the line height
\newcommand{\chuhao}{\fontsize{42pt}{42pt}\selectfont}
\newcommand{\xiaochu}{\fontsize{36pt}{36pt}\selectfont}
\newcommand{\yihao}{\fontsize{26pt}{26pt}\selectfont}
\newcommand{\xiaoyi}{\fontsize{24pt}{24pt}\selectfont}
\newcommand{\erhao}{\fontsize{22pt}{22pt}\selectfont}
\newcommand{\xiaoer}{\fontsize{18pt}{18pt}\selectfont}
\newcommand{\sanhao}{\fontsize{16pt}{16pt}\selectfont}
\newcommand{\xiaosan}{\fontsize{15pt}{15pt}\selectfont}
\newcommand{\sihao}{\fontsize{14pt}{14pt}\selectfont}
\newcommand{\xiaosi}{\fontsize{12pt}{12pt}\selectfont}
\newcommand{\wuhao}{\fontsize{10.5pt}{10.5pt}\selectfont}
\newcommand{\xiaowu}{\fontsize{9pt}{9pt}\selectfont}
\newcommand{\liuhao}{\fontsize{7.5pt}{7.5pt}\selectfont}
\newcommand{\xiaoliu}{\fontsize{6.5pt}{6.5pt}\selectfont}
\newcommand{\qihao}{\fontsize{5.5pt}{5.5pt}\selectfont}
\newcommand{\bahao}{\fontsize{5pt}{5pt}\selectfont}

\newcommand{\initfont}{                     % command for set font size and line height
    \renewcommand{\baselinestretch}{1.0}
    \fontsize{12pt}{20pt}\selectfont        % normal text
    \setlength{\parindent}{2em}             % normal paragraph indent
}

%% Contents styling
\renewcommand{\contentsname}{目\qquad 录}
\setcounter{tocdepth}{2}
\titlecontents{chapter}[0em]{\vspace{0pt}\xiaosi\songti}%
             {第 \thecontentslabel 章 \quad}{} %
             {\hspace{.5em}\titlerule*[5pt]{.}\xiaosi\contentspage}
\titlecontents{section}[2em]{\vspace{0pt}\xiaosi\songti} %
            {\thecontentslabel\quad}{} %
            {\hspace{.5em}\titlerule*[5pt]{.}\xiaosi\contentspage}
\titlecontents{subsection}[4em]{\vspace{0pt}\xiaosi\songti} %
            {\thecontentslabel\quad}{} %
            {\hspace{.5em}\titlerule*[5pt]{.}\xiaosi\contentspage}

%% ReDefine the chapters and headings style

% 一般分为章标题(一级标题)、不编号章标题(一级标题)、二级标题、三级标题和四级标题。标题编号与标题内容之间键入1个空格。
% 章标题(一级标题):章标题采用黑体小三号字。段前、段后间距在30~36磅之间(含)。章标题居中显示,编号为阿拉伯数字(形如“第1章”)。
% 不编号章标题(一级标题):不编号章标题要求同章标题但无编号。
% 二级标题:二级标题采用黑体四号字。段前、段后间距在18~24磅之间(含)。二级标题应该无缩进左对齐。编号应包含一级标题,且用英文句号连接。
% 三级标题:三级标题采用黑体四号字。段前、段后间距在12~15磅之间(含)。三级标题应该无缩进左对齐。编号应包含一级标题和二级标题,且用英文句号连接。
% 四级标题:四级标题采用黑体小四号字。段前、段后间距在9~12磅之间(含)。四级标题应该无缩进左对齐。编号应包一级标题、二级标题和三级标题,且用英文句号连接。

\setcounter{secnumdepth}{4} % This command sets the depth of section numbering to 4 levels. This means that all sections, including subsubsection, will be numbered.
\renewcommand{\chaptername}{第 \thechapter 章}
\titleformat{\chapter}{\centering\xiaosan \heiti}{\chaptername}{1em}{}
\titlespacing{\chapter}{0pt}{0pt}{30pt} % This command sets the spacing around chapter titles. There will be 30pt of space above and below the chapter title, and no horizontal spacing.
% \documentclass{book}中本身含有Chapter的下移的定义,所以标题上间距量填30会导致上方间距巨大。下方级别标题有自带的下偏移量,同理。
\titleformat*{\chapter}{\centering\xiaosan\heiti}   % style for the unnumbered chapters
\titleformat{\section}{\sihao\heiti}{\thesection}{1em}{}
\titlespacing{\section}{0pt}{18pt}{15pt}
\titleformat{\subsection}{\sihao\heiti}{\thesubsection}{1em}{}
\titlespacing{\subsection}{0pt}{12pt}{9pt}
\titleformat{\subsubsection}{\xiaosi\heiti}{\thesubsubsection}{1em}{}
\titlespacing{\subsubsection}{0pt}{9pt}{6pt}
\newcommand{\bigchapter}[1]{\chapter*{\centering \songti \erhao \textbf{#1}}}   % style for the abstracts.


%% Define headers and footers
\pagestyle{fancy}               % fancy is to show the header and footer
\renewcommand{\chaptermark}[1]{%
    \markboth{\chaptername~\ #1}{}}
\fancyhf{}%
\fancyhead[CO]{\songti \wuhao \leftmark}    % the CO means centering the odd pages' header
\fancyhead[CE]{\songti \wuhao 天津大学硕士学位论文} % the CE centering the even pages' header
\fancyfoot[C]{\songti \xiaowu ~\thepage~}   % the C means centering the footer
\renewcommand{\headrulewidth}{0.5pt}        % the width of the header line
\renewcommand{\footrulewidth}{0pt}          % the width of the footer line
\setlength{\footskip}{20pt}                 % the distance between the bottom of the page and the footer

\fancypagestyle{plain}{%        % plain is to show the footer only
    \fancyhf{}                  % clear all header and footer fields
    \fancyfoot[C]{\songti \xiaowu ~\thepage~ }  % show the page number in the center of footer
    \renewcommand{\headrulewidth}{0pt}          % clear the header line
    \renewcommand{\footrulewidth}{0pt}          % clear the footer line
}

%% references name
\renewcommand{\bibname}{参考文献}

%% Table/figure/equations captions
% 文中的图、表、公式一律采用阿拉伯数字分章编号。若图或表中有附注,采用英文小写字母顺序编号。子图采用英文字母编号
% for figures and tables
\renewcommand{\tablename}{表}
\renewcommand{\thetable}{\arabic{chapter}-\arabic{table}}
\renewcommand{\figurename}{图}
\renewcommand{\thefigure}{\arabic{chapter}-\arabic{figure}}
\renewcommand{\thesubfigure}{(\alph{subfigure})}
\renewcommand{\thesubtable}{(\alph{subtable})}
\newcommand{\tabfont}{\fontsize{10.5}{20}\selectfont}     % default table style 五号,固定行距20pt
\captionsetup{labelsep=space}   % remove the default colon between numbering and texts, remove this line and the caption will become 表1-1: 表标题 or 图1-1: 图表题
\captionsetup{font=singlespacing,labelsep=space}% caption seperator: space. E.g 表1-1 表标题 or 图1-1 图表题
\captionsetup[subfloat]{labelformat=simple}
\renewcommand{\captionfont}{\fontsize{12pt}{20pt}\selectfont}
% 图和表的标题规定中并未说明具体标题文字要求和字号要求,只规定了内容字体,那标题就认为是正文格式叭。
% 图序及图题居中置于图的下方,表序及表题居中置于表的上方。
\captionsetup[table]{position=top,skip=0pt}
\captionsetup[figure]{position=bottom,skip=0pt,belowskip=-10.5pt}
% 图、表等与其前后的正文之间要有一行的间距
\textfloatsep = 20pt plus 0pt minus 10.5pt
\floatsep = 10.5pt plus 0pt minus 10.5pt
\intextsep= 20pt plus 0pt minus 10.5pt

% for equations
\renewcommand{\theequation}{\arabic{chapter}-\arabic{equation}}
\abovedisplayshortskip=0pt
\belowdisplayshortskip=0pt
\abovedisplayskip=10.5pt
\belowdisplayskip=10.5pt











