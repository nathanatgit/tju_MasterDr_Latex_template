\chapter{示例章节}

本页面为图文、公式、表格混排示例。

\section{一级标题}
\thispagestyle{fancy}
《元日》·李世民〔唐代〕。
高轩暧春色,邃阁媚朝光。
彤庭飞彩旆,翠幌曜明珰。
恭己临四极,垂衣驭八荒。
霜戟列丹陛,丝竹韵长廊。

\begin{table}[!h]
    \centering
    \caption{表格标题}
    \label{tab:exp1}
    \tabfont % 激活预定义的默认表格文字和行间距大小
    \begin{tabular}{cc}
        \toprule
        第一列                     & 第二列                     \\
        \midrule
        图与其前后的正文之间要有一 & 正文之间要有               \\
        3                          & 图、表等与其前后的正文之间 \\
        \bottomrule
    \end{tabular}
\end{table}

穆矣熏风茂,康哉帝道昌。
继文遵后轨,循古鉴前王。
草秀故春色,梅艳昔年妆。
巨川思欲济,终以寄舟航。

\subsection{二级标题}
《陋室铭》·刘禹锡〔唐代〕。山不在高,有仙则名。水不在深,有龙则灵。斯是陋室,惟吾德馨。苔痕上阶绿,草色入帘青。谈笑有鸿儒,往来无白丁。可以调素琴,阅金经。无丝竹之乱耳,无案牍之劳形。南阳诸葛庐,西蜀子云亭。孔子云:何陋之有?

表格交叉引用示例,如表 \ref{tab:exp1}。

\begin{figure}[!h]
    \centering
    \includegraphics[width= 6cm]{example-image-golden}
    \caption{这是一个图片标题.}
    \label{fig:exp1}
\end{figure}

译文:山不在于高,有了神仙就会有名气。水不在于深,有了龙就会有灵气。这是简陋的房子,只是我品德好就感觉不到简陋了。苔痕碧绿,长到台上,草色青葱,映入帘里。到这里谈笑的都是博学之人,来往的没有知识浅薄之人,可以弹奏不加装饰的琴,阅读佛经。没有弦管奏乐的声音扰乱耳朵,没有官府的公文使身体劳累。南阳有诸葛亮的草庐,西蜀有扬子云的亭子。孔子说:有什么简陋的呢?

图交叉引用示例,如图 \ref{fig:exp1}。

\subsubsection{三级标题}

铭是古代一种刻于金石上的押韵文体,多用于歌功颂德与警戒自己。明白了铭的意思,也就明白了题意,作者托物言志,通过对居室的描绘,极力形容陋室的不陋,“斯是陋室,惟吾德馨”这一中心,实际上也就是借陋室之名行歌颂道德品质之实,表达出室主人高洁傲岸的节操和安贫乐道的情趣。
\begin{equation}
    % 公式贴着上一段写,要不然上间距会很宽
    \label{eqn:y1}
    y = hx+n
\end{equation}

《陋室铭》即开篇以山水起兴,山可以不用高,水可以不在深,只要有了仙龙就可以出名,那么居所虽然简陋,但却因主人的有“德”而“馨”,也就是说陋室因为有道德品质高尚的人存在当然也能出名,声名远播,刻金石以记之。山水的平凡因仙龙而生灵秀,那么陋室当然也可借道德品质高尚之士播洒芬芳。此种借力打力的技巧,实为绝妙,也可谓作者匠心独具。特别是以仙龙点睛山水,构思奇妙。“斯是陋室,唯吾德馨”,由山水仙龙入题,作者笔锋一转,直接切入了主题,看引论铺下了基础。也点出了陋室不陋的原因,其原因是德馨二字。
\begin{equation}
    \label{eqn:barwq}
    \begin{split}
        H_c&=\frac{1}{2n} \sum^n_{l=0}(-1)^{l}(n-{l})^{p-2}
        \sum_{l _1+\dots+ l _p=l}\prod^p_{i=1} \binom{n_i}{l _i}\\
        &\quad\cdot[(n-l )-(n_i-l _i)]^{n_i-l _i}\cdot
        \Bigl[(n-l )^2-\sum^p_{j=1}(n_i-l _i)^2\Bigr].
    \end{split}
\end{equation}

公式交叉引用示例,如公式 \eqref{eqn:y1}和\eqref{eqn:barwq}。

\section{一级标题}
《竹石》·郑燮 〔清代〕。
咬定青山不放松,立根原在破岩中。
千磨万击还坚劲,任尔东西南北风。

\subsection{二级标题}
译文:竹子抓住青山一点也不放松,它的根牢牢地扎在岩石缝中。
经历无数磨难和打击身骨仍坚劲,任凭你刮酷暑的东南风,还是严冬的西北风。

\subsubsection{三级标题}
这首诗着力表现了竹子那顽强而又执着的品质 。是一首赞美岩竹的题画诗,也是一首咏物诗。开头用“咬定”二字,把岩竹拟人化,已传达出它的神韵和它顽强的生命力;后两句进一步写岩竹的品格,它经过了无数次的磨难,才长就了一身英俊挺拔的身姿,而且从来不畏惧来自东西南北的狂风的击打。
\begin{table}[!h]
    \centering
    \caption{表格标题}
    \label{tab:exp2}
    \tabfont % 激活预定义的默认表格文字和行间距大小
    \begin{tabularx}{\textwidth}{XX}
        \toprule
        第一列                     & 第二列                     \\
        \midrule
        图与其前后的正文之间要有一 & 正文之间要有               \\
        3                          & 图、表等与其前后的正文之间 \\
        \bottomrule
    \end{tabularx}
\end{table}
郑燮不但写咏竹诗美,而且画出的竹子也栩栩如生,在他笔下的竹子竹竿很细,竹叶着色不多,却青翠欲滴,并全用水墨,更显得高标挺立,特立独行。所以这首诗表面上是写竹,实际上是写人,写作者自己那种正直、刚正不阿、坚强不屈的性格,决不向任何邪恶势力低头的高风傲骨。同时,这首诗也能给我们以生命的感动,曲折恶劣的环境中,战胜困难,面对现实,像在石缝中的竹子一样刚强勇敢,体现了爱国者的情怀。